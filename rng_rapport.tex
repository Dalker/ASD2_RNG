\documentclass{scrartcl}

% importations de packages utiles
\usepackage[backend=biber, style=alphabetic]{biblatex}
\addbibresource{rng.bib}
\usepackage[utf8]{inputenc}  % pouvoir écrire avec des accents
\usepackage[french]{babel}  % francophopnie
\usepackage{hyperref}  % liens clicables dans pdf final
\usepackage{tikz}  % pouvoir tracer des dessins sympas
\usepackage{listingsutf8}  % rendu de "code" (avec config ci-dessous)
\definecolor{lstcolor}{rgb}{0.9,0.95,0.95}
\definecolor{lstcommentcolor}{rgb}{0.,0.2,0.}
\lstset{
  frameround=tttt,
  %autogobble,
  frame=single,
  backgroundcolor=\color{lstcolor},
  % extendedchars=true,
  % basicstyle=\ttfamily\small,
  keywordstyle=\bfseries\color{blue},
  identifierstyle=\bfseries\color{red},
  stringstyle=\bfseries\color{orange},
  commentstyle=\color{lstcommentcolor},
  language=Python,
  keepspaces=True,
  basicstyle=\fontfamily{pcr}\selectfont\small, % monospace it for copypasting
  upquote=true,
  columns=flexible,
  showstringspaces=False,
  literate={é}{{\'e}}1
}
\title{Générateurs de nombres aléatoires}
\subtitle{Algorithmes et Structures de Données II, GymInf}
\author{Juan-Carlos Barros, Yves Dethurens, Daniel Kessler et Jean-Francis Ravoux}
% et c'est parti
\begin{document}
\maketitle

\tableofcontents

\section{Introduction}
\subsection{Que veut-on simuler et pourquoi?}
\begin{itemize}
\item distributions aléatoires (bla)
\item utilité directe (ex: jeus) et indirecte (ex: algos aléatoires)
\end{itemize}

\subsection{TRNG vs PRNG}
\ldots

\section{Générateurs de suites pseudo-aléatoires}
\subsection{Historique}
von Neumann\cite{VonNeumann} (bla)

\subsection{Caractéristiques communes}
\begin{itemize}
\item période (bla)
\item seed (bla)
\end{itemize}

\subsection{Générateurs linéaires congruents (LCG)}
\ldots

\subsection{Mersenne Twister et les LFSR}
\ldots

\section{Générateurs de ``vraies'' suites aléatoires}
\subsection{Collection d'entropie}
\ldots
\subsection{Algorithmes d'aggrégation d'entropie}
HAVEGE etc.
\subsection{Le futur est-il quantique?}
Gisin et co.

\section{Que fait le module ``random'' de Python?}
bla sur les PRNG et TRNG utilisés (indirectement) par Python et résumé de
résultats de petits tests

\section{Conclusion}
bla bla

\addcontentsline{toc}{section}{Références}
\printbibliography
\end{document}