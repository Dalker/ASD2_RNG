\documentclass{scrartcl}

% importations de packages utiles
\usepackage[backend=biber, style=alphabetic]{biblatex}
\addbibresource{rng.bib}
\usepackage[utf8]{inputenc}  % pouvoir écrire avec des accents
\usepackage[french]{babel}  % francophopnie
\usepackage{hyperref}  % liens clicables dans pdf final
\usepackage{tikz}  % pouvoir tracer des dessins sympas
\usepackage{listingsutf8}  % rendu de "code" (avec config ci-dessous)
\definecolor{lstcolor}{rgb}{0.9,0.95,0.95}
\definecolor{lstcommentcolor}{rgb}{0.,0.2,0.}
\lstset{
  frameround=tttt,
  %autogobble,
  frame=single,
  backgroundcolor=\color{lstcolor},
  % extendedchars=true,
  % basicstyle=\ttfamily\small,
  keywordstyle=\bfseries\color{blue},
  identifierstyle=\bfseries\color{red},
  stringstyle=\bfseries\color{orange},
  commentstyle=\color{lstcommentcolor},
  language=Python,
  keepspaces=True,
  basicstyle=\fontfamily{pcr}\selectfont\small, % monospace it for copypasting
  upquote=true,
  columns=flexible,
  showstringspaces=False,
  literate={é}{{\'e}}1
}
\title{Générateurs de nombres aléatoires}
\subtitle{Algorithmes et Structures de Données II, GymInf}
\author{Juan-Carlos Barros, Yves Dethurens, Daniel Kessler et Jean-Francis Ravoux}
% et c'est parti
\begin{document}
\maketitle

\tableofcontents

\section{Introduction}
\subsection{Que veut-on simuler et pourquoi?}
\begin{itemize}
\item distributions aléatoires (bla)
\item utilité directe (ex: jeus) et indirecte (ex: algos aléatoires)
\end{itemize}

\subsection{TRNG vs PRNG}
\ldots

\section{Générateurs de suites pseudo-aléatoires}
\subsection{Historique}
von Neumann\cite{VonNeumann} (bla)

\subsection{Caractéristiques communes}
\begin{itemize}
\item période (bla)
\item seed (bla)
\end{itemize}

\subsection{Générateurs linéaires congruents (LCG)}
\ldots

\subsection{Mersenne Twister et les LFSR}
\ldots

\section{Générateurs de ``vraies'' suites aléatoires}
\subsection{Généralités - Processeur incapable}
 \begin{itemize}
 \item Processeur arrive plutôt bien à propager de l’aléatoire
 \item Voir algorithmes présentés précédemment
 \item Mais il lui faut un coup de pouce au départ
 \item  Besoin d’une graine pour démarrer
 \item Pourquoi hasard inaccessible au processeur?
 \item Car le processeur est profondément déterministe
 \end{itemize}

\subsection{Généralités - Le Monde réel oui}
 \begin{itemize}
 \item Aléatoire inévitable et dérangeant dans le monde réel!
  \begin{itemize}
  \item Incertitudes fondamentales des mesures
  \item Impossibilité de contrôler une valeur physique
  \end{itemize}
 \item Monde réel est donc LA source d’inspiration
 \end{itemize}

\subsection{Collection d'entropie}
 \begin{itemize}
 \item Principales sources de hasard:
  \begin{itemize}
  \item phénomènes physiques stochastiques:
   \begin{itemize}
   \item bruit thermique (Johnson et Nyquist)
   \item autres phénomènes statistiques (vagues, etc.)
   \end{itemize}
  \item phénomènes quantiques intrinsèquement aléatoires
   \begin{itemize}
   \item effet photoélectrique
   \item n’importe quelle autre mesure quantique
   \end{itemize}
  \end{itemize}
 \end{itemize}


\subsection{Algorithmes d'aggrégation et expansion d'entropie}
 \begin{itemize}
 \item Algorithme pour grossir le flux de TRNG (pas assez rapide)
 \item HAVEGE (utilisé pas le noyau Linux)
 \item HArdware Volatile Entropy Gathering and Expansion
 \item https://www.irisa.fr/caps/projects/hipsor/misc.php
 \end{itemize}

\subsection{Le futur est-il quantique?}
 \begin{itemize}
 \item Sources quantiques:
  \begin{itemize}
  \item source de radioactivité détectée par un compteur Geiger
  \item photons traversant un miroir semi-réfléchissant
  \item C’est le choix de la compagnie Genevoise ID Quantique
  \end{itemize}
 \end{itemize}

\subsection{Exemple genevois: ID Quantique}
 \begin{itemize}
 \item Principe de la source ID Quantique:
  \begin{itemize}
  \item photons traversant un miroir semi-réfléchissant
  \item événements mutuellement exclusifs (réflexion / transmission)
  \item Détection associée respectivement à des valeurs de bit 0 ou 1
  \end{itemize}
 \end{itemize}

\section{Que fait le module ``random'' de Python?}
bla sur les PRNG et TRNG utilisés (indirectement) par Python et résumé de
résultats de petits tests

\section{Conclusion}
bla bla

\addcontentsline{toc}{section}{Références}
\printbibliography
\end{document}