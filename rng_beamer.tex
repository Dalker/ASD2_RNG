\documentclass{beamer}

% importations de packages utiles
\usepackage[utf8]{inputenc}  % pouvoir écrire avec des accents
\usepackage[french]{babel}  % francophopnie
\usepackage{hyperref}  % liens clicables dans pdf final
\usepackage{tikz}  % pouvoir tracer des dessins sympas
\usetheme{Boadilla}  % thème de beamer
\usepackage{listingsutf8}  % rendu de "code" (avec config ci-dessous)
\definecolor{lstcolor}{rgb}{0.9,0.95,0.95}
\definecolor{lstcommentcolor}{rgb}{0.,0.2,0.}
\lstset{
  frameround=tttt,
  %autogobble,
  frame=single,
  backgroundcolor=\color{lstcolor},
  % extendedchars=true,
  % basicstyle=\ttfamily\small,
  keywordstyle=\bfseries\color{blue},
  identifierstyle=\bfseries\color{red},
  stringstyle=\bfseries\color{orange},
  commentstyle=\color{lstcommentcolor},
  language=Python,
  keepspaces=True,
  basicstyle=\fontfamily{pcr}\selectfont\small, % monospace it for copypasting
  upquote=true,
  columns=flexible,
  showstringspaces=False,
  literate={é}{{\'e}}1
}
\title{Projet \textit{RNG}}
\subtitle{Algorithmes et Structures de Données II}
\author{Juan-Carlos Barros, Yves Dethurens, Daniel Kessler et Jean-Francis Ravoux}
% et c'est parti
\begin{document}
\begin{frame}
  \titlepage
\end{frame}

\begin{frame}
  \tableofcontents
\end{frame}

\section{Distributions}
\begin{frame}
  \frametitle{Distributions aléatoires}
  \begin{tikzpicture}
    \draw[very thin,color=black] (-0,-0) grid (4,4);
    \draw[->] (-0.2,0) -- (4.2,0) node[right] {$x$};
    \draw[->] (0,-1.2) -- (0,4.2) node[above] {$f(x)$};
    \draw[blue] (0,0) plot[domain=-4:4] (\x,{2^{-\x*\x}}) node[anchor=south west] {$V_{p}$};
  \end{tikzpicture}
\end{frame}

\section{``vrai'' ou ``pseudo''-aléatoire?}
\begin{frame}
  \frametitle{``vrai'' ou ``pseudo''-aléatoire?}
\end{frame}
\end{document}