\documentclass{beamer}

% tout le bazar jusqu'à title, c'est du chinois beamer/pdf illisible...

% importations de packages utiles
\usepackage[utf8]{inputenc}  % pouvoir écrire avec des accents
\usepackage[french]{babel}  % francophopnie
\usepackage{hyperref}  % liens clicables dans pdf final
\usepackage{tikz}  % pouvoir tracer des dessins sympas
\usepackage{graphicx}

\usetheme{Boadilla}  % thème de beamer
\usepackage{listingsutf8}  % rendu de "code" (avec config ci-dessous)
\definecolor{lstcolor}{rgb}{0.9,0.95,0.95}
\definecolor{lstcommentcolor}{rgb}{0.,0.2,0.}
\lstset{
  frameround=tttt,
  %autogobble,
  frame=single,
  backgroundcolor=\color{lstcolor},
  % extendedchars=true,
  % basicstyle=\ttfamily\small,
  keywordstyle=\bfseries\color{blue},
  identifierstyle=\bfseries\color{red},
  stringstyle=\bfseries\color{orange},
  commentstyle=\color{lstcommentcolor},
  language=Python,
  keepspaces=True,
  basicstyle=\fontfamily{pcr}\selectfont\small, % monospace it for copypasting
  upquote=true,
  columns=flexible,
  showstringspaces=False,
  literate={é}{{\'e}}1
}
\title{Projet \textit{RNG}}
\subtitle{Algorithmes et Structures de Données II}
\author{Juan-Carlos Barros, Yves Dethurens, Daniel Kessler et Jean-Francis Ravoux}
% et c'est parti
\begin{document}

% \section{Exemple pour Jean-Francis}
% \begin{frame}{Exemple pour Jean-Francis TEXTE}
%   \begin{minipage}{\textwidth}
%     \onslide<1->{Du texte 1 Du texte 1Du texte 1Du texte 1Du texte 1Du texte 1\par}
%     \onslide<2->{Du texte 2 Du texte 2Du texte 2Du texte 2Du texte 2Du texte 2}
%   \end{minipage}
%   \begin{minipage}{0.33\textwidth}
%     Du texte inutile à gauche
%   \end{minipage}

%     \begin{minipage}{0.32\textwidth}
%     \only<1-2>{Du texte inutile milieu PROBLEME}
%     \only<3->{Un texte différent}
%   \end{minipage}

% \end{frame}


  % \begin{minipage}{0.67\textwidth}
    % \only<2>{Cas 1: X_n \in [0,2,3,1] période p = 4}
  %   \onslide<3->{Cas 2: X_n \in [0,2,1,6,8,7,3,5,4]\par
  %   période p = 9}
  %   \onslide<4->{Maximiser les nombres aléatoires\par
  %   module m grand\\
  %   conditions sur a et c ("magic values)}
  % \end{minipage}

\section{Les Générateurs de Congruence Linéaire}
\begin{frame}{Les Générateurs de Congruence Linéaire}
  \begin{minipage}{\textwidth}
    \only<1,4,5>{\begin{figure}
      \includegraphics[scale=1]{GCL1-général.png}
      \end{figure}\par}
    \only<2>{\begin{figure}
      \includegraphics[scale=1]{GCL1-val1.png}
      \end{figure}\par}
    \only<3>{\begin{figure}
      \includegraphics[scale=1]{GCL1-val2-magic.png}
      \end{figure}\par}
    
  \end{minipage}

  \begin{minipage}{0.32\textwidth}
    \onslide<1->{[m] module m \\
    a multiplicateur \\
    c incrément \\
    $X_0$ graine \\
    $X_{n} \in \{0, .., m-1\}$\par}
  \end{minipage}
  % \begin{minipage}{0.33\textwidth}
  %   Du texte inutile
  % \end{minipage}
  \begin{minipage}{0.67\textwidth}
    \only<2>{Cas 1: $X_n \in [1,0,2,3,1]$ \\ période p = 4}
    \only<3>{Cas 2: $X_n \in [4,0,2,1,6,8,7,3,5,4]$\\ période p = 9}
    \only<4>{Cas 1: (a=3,c=2,m=5) mauvaises valeurs \\ Cas 2: (a=4,c=2,m=9) valeurs magiques\\
    conditions pour optimiser les nombres aléatoires:\\
    module grand\\
    conditions sur a et c pour maximiser p
    }
    % \only<5>{Critères:\\
    % PGCD(a,m)=1;PGCD(c,m)=1\\
    % m multiple d'un nombre premier alors a-1 est multiple de ce nombre\\
    % a-1 est multiple de 4 si m est multiple de 4 }
    \only<5>{
    RANDU (IBM) $X_{n+1} =65539 X_{n} [2^{31}]$\\
    Turbo Pascal $X_{n+1} =(129 X_{n} + 907633385)[2^{32}]$\\
    Unix $X_{n+1} =(1103515245 X_{n}+12345) [2^{31}]$}
  \end{minipage}

\end{frame}

\end{document}